%%%%%%%%%%%%%%%%%%%%%%%%%%%%%%%%%%%%%%%%%%%%%%%%%%%%%%%%%%%%%%%%%%%%%
%
% CS484 Written Question Template
%
% This is a LaTeX document. LaTeX is a markup language for producing
% documents. Your task is to fill out this document, then to compile
% it into a PDF document.
%
%
% TO COMPILE:
% > pdflatex thisfile.tex
%
% If you do not have LaTeX and need a LaTeX distribution:
% - Personal laptops (all common OS): www.latex-project.org/get/
% - We recommend miktex (https://miktex.org/) for latex engine,
%   and TeXstudio(http://www.texstudio.org/) for latex editor.
%   You should install both programs for editing latex.
%   Or you can use Overleaf (https://www.overleaf.com/) which is
%   an online latex editor.
%
% If you need help with LaTeX, please come to office hours.
% Or, there is plenty of help online:
% https://en.wikibooks.org/wiki/LaTeX
%
% Good luck!
% Min and the CS484 staff
%
%%%%%%%%%%%%%%%%%%%%%%%%%%%%%%%%%%%%%%%%%%%%%%%%%%%%%%%%%%%%%%%%%%%%%

\documentclass[11pt]{article}

\usepackage[english]{babel}
\usepackage[utf8]{inputenc}
\usepackage[colorlinks = true,
            linkcolor = blue,
            urlcolor  = blue]{hyperref}
\usepackage[a4paper,margin=1.5in]{geometry}
\usepackage{stackengine,graphicx}
\usepackage{fancyhdr}
\setlength{\headheight}{15pt}
\usepackage{microtype}
\usepackage{times}

% From https://ctan.org/pkg/matlab-prettifier
\usepackage[numbered,framed]{matlab-prettifier}

\frenchspacing
\setlength{\parindent}{0cm} % Default is 15pt.
\setlength{\parskip}{0.3cm plus1mm minus1mm}

\pagestyle{fancy}
\fancyhf{}
\lhead{Homework 1 Questions}
\rhead{CS484}
\rfoot{\thepage}

\date{}

\title{\vspace{-1cm}Homework 1 Questions}


\begin{document}
\maketitle
\vspace{-2cm}
\thispagestyle{fancy}

\section*{Instructions}
\begin{itemize}
  \item Compile and read through the included MATLAB tutorial.
  \item 2 questions.
  \item Include code.
  \item Feel free to include images or equations.
  \item Please make this document anonymous.
  \item \textbf{Please use only the space provided and keep the page breaks.} Please do not make new pages, nor remove pages. The document is a template to help grading.
  \item If you really need extra space, please use new pages at the end of the document and refer us to it in your answers.
\end{itemize}


\section*{Submission}
\begin{itemize}
    \item Please zip your folder with \textbf{hw1\_student id\_name.zip} $($ex: hw1\_20201234\_Peter.zip$)$
    \item Submit your homework to \href{http://klms.kaist.ac.kr/course/view.php?id=109597}{KLMS}.
    \item An assignment after its original due date will be degraded from the marked credit per day: e.g., A will be downgraded to B for one-day delayed submission.
\end{itemize}

\pagebreak


\section*{Questions}


%%%%%%%%%%%%%%%%%%%%%%%%%%%%%%%%%%%

% Please leave the pagebreak
\paragraph{Q1:} We wish to set all pixels that have a brightness of 10 or less to 0, to remove sensor noise. However, our code is slow when run on a database with 1000 grayscale images.

\emph{Image:} \href{grizzlypeakg.png}{grizzlypeakg.png}

\begin{lstlisting}[style=Matlab-editor]
A = imread('grizzlypeakg.png');
[m1,n1] = size( A );
for i=1:m1
    for j=1:n1
        if A(i,j) <= 10
            A(i,j) = 0;
        end
    end
end
\end{lstlisting}

\paragraph{Q1.1:} How could we speed it up?

%%%%%%%%%%%%%%%%%%%%%%%%%%%%%%%%%%%
\paragraph{A1.1:}
We can use logical indexing instead of manually modifying the values in a for loop, like this:
\begin{lstlisting}[style=Matlab-editor]
img1 = imread('grizzlypeakg.png');
img1(img1 <= 10) = 0;
\end{lstlisting}


%%%%%%%%%%%%%%%%%%%%%%%%%%%%%%%%%%%

% Please leave the pagebreak
\pagebreak
\paragraph{Q1.2:} What factor speedup would we receive over 1000 images? Please measure it.

Ignore file loading; assume all images are equal resolution; don't assume that the time taken for one image $\times1000$ will equal $1000$ image computations, as single short tasks on multitasking computers often take variable time.

%%%%%%%%%%%%%%%%%%%%%%%%%%%%%%%%%%%
\paragraph{A1.2:}
I've measured the individual durations and printed out the average time for each of the methods with the follwing code:

\begin{lstlisting}[style=Matlab-editor]
time1_sum = 0;
time2_sum = 0;

for n = 1:1000
    img1 = imread('grizzlypeakg.png');
    img2 = imread('grizzlypeakg.png');

    tic;
    img1(img1 <= 10) = 0;
    t1 = toc;
    time1_sum = time1_sum + t1;

    tic;
    [m1, n1] = size(img2);
    for i = 1:m1
        for j = 1:n1
            if img2(i, j) <= 10
                img2(i, j) = 0;
            end
        end
    end
    t2 = toc;
    time2_sum = time2_sum + t2;
end

disp(time1_sum / 1000)
disp(time2_sum / 1000)
\end{lstlisting}

The results have shown an average of 0.0044 seconds for the logical indexing method, while the for loop method have shown an average of 0.0112 seconds. The former method was 2.54 times faster than the latter.
%%%%%%%%%%%%%%%%%%%%%%%%%%%%%%%%%%%

% Please leave the pagebreak
\pagebreak
\paragraph{Q1.3:} How might a speeded-up version change for color images? Please measure it.

\emph{Image:} \href{grizzlypeak.jpg}{grizzlypeak.jpg}

%%%%%%%%%%%%%%%%%%%%%%%%%%%%%%%%%%%
\paragraph{A1.3:}
A modified version of the code from Q1.2 yields an average of 0.0153 seconds for the logical indexing method and 0.0523 seconds for the for loop method. The former method was 3.42 times faster than the latter method.

Modified Code:
\begin{lstlisting}[style=Matlab-editor]
time1_sum = 0;
time2_sum = 0;

for n = 1:1000
    img1 = imread('grizzlypeak.jpg');
    img2 = imread('grizzlypeak.jpg');

    tic;
    img1(img1 <= 10) = 0;
    t1 = toc;
    time1_sum = time1_sum + t1;

    tic;
    [m1, n1, z] = size(img2);
    for i = 1:m1
        for j = 1:n1
            for k = 1:z
                if img2(i, j, k) <= 10
                    img2(i, j, k) = 0;
                end
            end
        end
    end
    t2 = toc;
    time2_sum = time2_sum + t2;
end

disp(time1_sum / 1000)
disp(time2_sum / 1000)
\end{lstlisting}


%%%%%%%%%%%%%%%%%%%%%%%%%%%%%%%%%%%

% Please leave the pagebreak
\pagebreak
\paragraph{Q2:} We wish to reduce the brightness of an image but, when trying to visualize the result, all we sees is white with some weird ``corruption'' of color patches.

\emph{Image:} \href{gigi.jpg}{gigi.jpg}

\begin{lstlisting}[style=Matlab-editor]
I = double( imread('gigi.jpg') );
I = I - 20;
imshow( I );
\end{lstlisting}

\paragraph{Q2.1:} What is incorrect with this approach? How can it be fixed while maintaining the same amount of brightness reduction?

%%%%%%%%%%%%%%%%%%%%%%%%%%%%%%%%%%%
\paragraph{A2.1:}
Line number 1 shows an incorrect way of converting image pixel values into double precision floating point numbers. Rather, the code should be:
\begin{lstlisting}[style=Matlab-editor]
I = im2double( imread('gigi.jpg') );
I = I - 20 / 255;
imshow( I );
\end{lstlisting}
We subtract 20 / 255 rather than 20 since the pixel values now range from 0 to 1, whereas the original pixel values ranged from 0 to 255. Therefore we scale the amount of change by dividing it by 255.

%%%%%%%%%%%%%%%%%%%%%%%%%%%%%%%%%%%

% Please leave the pagebreak
\pagebreak
\paragraph{Q2.2:} Where did the original corruption come from? Which specific values in the original image did it represent?

%%%%%%%%%%%%%%%%%%%%%%%%%%%%%%%%%%%
\paragraph{A2.2:}
Calling double() on the image would convert the pixel values into the double data type. Then calling the imshow() function to display the image confuses the function to think that since the pixel values are floating point values, they would range from 0 to 1, and not 0 to 255. Since the pixel values are actually in the 0 to 255 range and the imshow() function interprets any value over 1 to be 1, most of them appear to be white. One can see that some areas of the image have color, and those are the areas that are relatively darker and manage to be between 0 and 1 after subtracting 20 from their original value.


%%%%%%%%%%%%%%%%%%%%%%%%%%%%%%%%%%%

\end{document}
